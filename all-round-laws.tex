% Rules for the HuroCup All-Round Competition
% Jacky Baltes <jacky@cs.umanitoba.ca> 

\documentclass[12pt]{hurocup}
\begin{document}

\newcommand{\thisyear}{2010}

\newcommand{\HuroCup}{\textsc{HuroCup}}


\title{\HuroCup: All-Round Competition\\
  Laws of the Game \thisyear}


\author{Jacky Baltes\\
Autonomous Agents Laboratory\\
University of Manitoba\\
Winnipeg, Manitoba\\
Canada, R3T 2N2\\
Email: jacky@cs.umanitoba.ca\\
WWW: http://www.cs.umanitoba.ca/\~{ }jacky\\[5mm]
Kuo-Yang Tu\\
National Kaohsiung First University of Science and Technology\\
Kaohsiung City, R. O. C.\\
Email: tuky@ccms.nkfust.edu.tw\\
}

\maketitle
\begin{abstract}
The following rules and regulations govern the all-round event in
\HuroCup, a robotic game and robotics benchmark problem for humanoid
robots.
%
\end{abstract}

\section*{Latest Version of the Rules for \HuroCup}
\label{sec:updates}

The latest official version of the rules of the game for \HuroCup\ is
always available from the FIRA \HuroCup\ website (http://www.fira.net).

\section*{Changes to the Rules of \HuroCup\ Allround for \thisyear}

The order of the events used as a tie breaker was changed. The first
event used in case of a tie is the marathon run.

\newpage

\section{The All-Round Competition}
\label{sec:all-round} 

This section contains information about the all-round humanoid robot
competition as part of \HuroCup. The all-round competition is the most
important \HuroCup\ event as it tests the versatility of a humanoid
robot. 

The winner in the all-round competition is determined by the single
robot with the most points over all eight \HuroCup\ events: robot
dash, penalty kick, obstacle run, lift and carry, weight lifting,
marathon, basketball, and climbing wall.

\law[ALR]{Number of Robots}

\begin{lawlist}[ALR]
 \item A single robot competes in a match.
\end{lawlist}

\law[ALR]{The Players}

Please refer to the general \HuroCup\ laws for a description of
the players.

\begin{lawlist}[ALR]
  
\item A single robot has to perform all events to be eligible to
 compete in the all-round competition.
  
\item A team is not allowed to modify the robot in any way for the
 different events.

\item A team is allowed to repair their robot only after having
 received permission from the referee.
  
\end{lawlist}

\law[ALR]{The Referee}

Please refer to the general \HuroCup\ laws for a description of
the referee.

\law[ALR]{The Assistant Referee}

Please refer to the general \HuroCup\ laws for a description of
the assitant referee.

\law[ALR]{Game Play}

\begin{lawlist}[ALR]

\item The score of a robot in the all-round competition is the sum of
 all the robot's scores in the \HuroCup\ events.

\end{lawlist}

\law[ALR]{Method of Scoring}

\begin{lawlist}[ALR]

\item All robots that have not gained at least one point in any event
 are automatically awarded no rank and $0$ points.

\item The robot with the maximum sumary score over all \HuroCup\
 events is declared the winner and the other ranks are determined
 according on the summary score for the other robots.

\item In case of multiple robots with the same summary score, the
  maximum score of a robot in a single event is used as a tie breaker.

\item In case of multiple robots with the same summary score and the
  same maximum single event score, the score of the robot over a
  single event is used as a tie breaker. The order of events is
  marathon, lift and carry, obstacle run, climbing wall, basketball,
  weight lifting, penalty kick, and robot dash.

\item In contrast to the other \HuroCup\ events, there is only a
 single round in the all-round competition. 

\end{lawlist}

\end{document}
