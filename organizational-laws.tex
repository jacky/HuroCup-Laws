% Rules for the HuroCup Organization
% Jacky Baltes <jacky@cs.umanitoba.ca> 

\documentclass[12pt]{hurocup}

\begin{document}

\title{\HuroCup: Organization\\
  Laws of the Game 2009}

\author{Jacky Baltes\\
Autonomous Agents Laboratory\\
University of Manitoba\\
Winnipeg, Manitoba\\
Canada, R3T 2N2\\
Email: jacky@cs.umanitoba.ca\\
WWW: http://www.cs.umanitoba.ca/\~{ }jacky\\[5mm]
Kuo-Yang Tu\\
National Kaohsiung First University of Science and Technology\\
Kaohsiung City, R. O. C.\\
Email: tuky@ccms.nkfust.edu.tw\\
}

\maketitle
\begin{abstract}
The following rules and regulations govern the organization of the
\HuroCup, a robotic game and robotics benchmark problem for humanoid
robots.
%
\end{abstract}

\section*{Latest Version of the Rules for \HuroCup}
\label{sec:updates}

The latest official version of the rules of the game for \HuroCup\ is
always available from the FIRA \HuroCup\ website (http://www.fira.net).

\newpage

\section{Organization of the \HuroCup\ Competition}
\label{sec:organization} 

This section contains information about the organization and the
running of the competition. These rules are not actually part of the
laws of the game, but rather specify the number of runs for events and
similar 

\law[ORG]{Number of Events}
\label{law:number-of-events}

\begin{lawlist}[ORG]
  
\item The local organizing committee determines the number of rounds
 for the various events (e.g., robot dash, penalty kick, weight lifting).
  
\item If at all possible, the rounds for different events should be spread out  over several days.
 
\item The local organizing committee determines whether if any of the rounds 
 for an event can be scratched. For example, the local organizing
 committee may decide to count only the top two scores out of three
 rounds for an event to calculate a team's score for a single
 event. The number of rounds that are counted towards the final score
 is called the score count $C$.

\end{lawlist}

\begin{decisions}
\item In the 2009 competition, an exhibition game of 3 vs. 3 soccer
will take place. All interested parties are encouraged to contact the
organzing chair.
\end{decisions}

\law[ORG]{The Free Program}

\begin{lawlist}[ORG]
  
\item At the end of the competition, all competitors take part in the
 ``free program.'' The aim of the ``free program'' is to provide an
 enjoyable show for the spectators as well as to allow the competitors
 to highlight specific features of their robot. It also allows the
 organizing committee to try out possible new events or variations for
 the next year.
  
\item The robots will perform events that are loosely based on the
 competition events. 
  
\item Every team will be allocated a maximum of three minutes to give
 a short demonstration highlighting some feature of their robot. For
 example, a robot with good balance could demonstrate this by walking
 over a balancing pole.
  
\item The ``free program'' does not count towards the final score of
 the robots, but the performance of the robots will be considered in
 the selection for the technical merit awards.

\end{lawlist}

\law[ORG]{Prizes and Awards}

\begin{lawlist}[ORG]
  
\item The final score for a robot in an event is calculated as the sum of 
 the top $C$ results, where $C$ is the score count determined by the local
 organizers.
  
\item The first prize is awarded to the robot with the maximum final score.
  
\item In case of a tie, the maximum score of a robot in any individual
  round is taken as a tie breaker.
  
\item In case of two robots having the same final score as well as the
 same maximum score in an individual round, the sum of the raw
 performances (e.g., time, distance, number of successful tries or
 weight) is used as a tie breaker.

\item There will be a place award for the first, second, and third
  placed robot in each event.

\end{lawlist}

\law[ORG]{Technical Merit Awards}

\begin{lawlist}[ORG]
  
\item There may be a maximum of two technical awards given out to the
  teams that made most significant contribution to research in
  humanoid robots. 
  
\item In general, one technical award is intended for technical
  contributions (mechanics), and the other for high level improvements
  (control, planning, and learning).
  
\item The intention of the technical awards is to reward contributions
  that have not been rewarded previously in the events of the
  competition. 
  
  For example, a robot would not receive an award for fastest robot
  would not be appropriate since running speed is the primary factor
  influencing the performance of a humanoid robot in the robot dash
  event. On the other hand, a robot demonstrating significant
  improvements in human robot interaction may receive a technical
  award.

\item The selection of the technical awards are done through the
  program committee.

\end{lawlist}

\end{document}
